\section{Estado del arte}

% En el punto de motivación y objetivos debes de dejar muy claro que es lo que quieres hacer (un detector de baches) y en este tema 2 debes de explicar como se solucionan este tipo de problemas (CNN) y las redes y frameworks existentes para poder realizar tu trabajo. Como esto ya es una cosa que has hecho (me lo mandaste en un mail muy bien explicado) debes de contar los pros y contras y justificar el porque has seleccionado la red seleccionada

El problema que se pretende resolver podría ser afrontado de dos posibles maneras:

\begin{itemize}
	\item Como un problema de clasificación de imágenes
	\item Como un problema de detección de objetos
\end{itemize}

El primero de los enfoques es más sencillo y está más estudiado. Dada una imagen, se determina una clase a la que pertenece la imagen. En los problemas de clasificación cada una de las imágenes se centran en un único objeto. Este tipo de problemas de clasificación se resuelven comúnmente con redes neuronales convolucionales. Existen numerosas arquitecturas de redes neuronales convolucionales ya definidas y estudiadas para resolver este tipo de problemas, como por ejemplo: VGG-16, LeNet, ResNet, GoogLeNet/Inception, etc.

El segundo de los enfoques es más complicado, y presenta varios retos. El primero de ellos es que las imágenes no se centran en un único objeto, sino que puede haber múltiples objetos a detectar y además tratarse de objetos de distintos tipos. El segundo de los restos es el tamaño de los objetos a identificar, que puede ser variable. Y el tercero de los retos es que se están resolviendo dos problemas al mismo tiempo: localizar objetos en una imagen y clasificar los objetos localizados.

Para resolver los problemas de detección de objetos existen dos aproximaciones. La primera de las aproximaciones es una aproximación clásica, basada en técnicas de machine learning. Un ejemplo representativo de esta aproximación clásica es Viola-Jones, que se basa en clasificadores binarios y que se ha usado en las cámaras de fotos para la detección de caras.

El uso del deep learning para la detección de objetos ha supuesto una revolución y ha cambiado las reglas del juego. Esta aproximación para la resolución de este tipo de problemas es relativamente reciente y ha estado en constante evolución.

{\color{red} \textbf{!!! TODO}}

\begin{itemize}
	\item R-CNN
	\item Fast R-CNN
	\item Faster R-CNN
	\item SDD
	\item YOLO (YOLO, YOLOv2, YOLOv3)
	\item Mask R-CNN
\end{itemize}
\begin{poliabstract}{Abstract}
\noindent
The state of asphalt on both national and urban roads is of high importance in relation to road safety. Currently, thanks to technological advances, automatic detection systems are being developed to detect potholes in the asphalt, allowing early detection of these irregularities on the road.

\doublespacing\singlespacing
\noindent
This project aims to contribute to this field by developing an automatic pothole detection system based on images. Starting from a set of images labeled with potholes, a neural network YOLO v3 and another YOLO v3 Tiny have been trained. The training has been carried out with different network sizes and with different subsets of images, resulting in fifteen models. After a comparative study of the precisions of the models, those with better results have been exported and transformed to be executable in a mobile device. Finally, an Android mobile application has been developed that loads the previously exported models and executes them with images obtained from the device's camera.

\doublespacing\singlespacing
\noindent
All the image pre-processing, neural network training, evaluation, export and transformation of the models has been carried out in the cloud on the Google Colab platform.
\end{poliabstract}
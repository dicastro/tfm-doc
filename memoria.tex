\documentclass[]{article}

%opening
\title{Sistema de detección automática de socavones en el asfalto a partir de imágenes}
\author{Diego Castro Viadero}
\date{Septiembre 2019}

\renewcommand*\contentsname{Contenido}

\begin{document}

\maketitle

\begin{abstract}
El estado del asfalto en carreteras tanto de ámbito nacional como de ámbito urbano es de alta importancia en relación a la seguridad vial. En la actualidad, no existe un sistema de detección automática de socavones en el asfalto. Tan sólo se tiene conocimineto de los mismos cuando han sido los causantes de un accidente vial o de una queja ciudadana (detección pasiva).

Este proyecto pretende desarrollar un sistema de detección automática y activa de socavones a partir de imágenes, que permita a las autoridades pertinentes conocer el número y ubicación de los mismos. Los principales objetivos son:

\begin{itemize}
	\item Detección temprana y activa de socavones a partir de imágenes
	\item Creación de una base de datos con la relación de socavones detectados (número y ubicación)
\end{itemize}

Los principales beneficios son:

\begin{itemize}
	\item Optimización de recursos necesarios para la reparación de socavones
	\item Aumentar la seguridad vial de las carreteras y evitar accidentes
	\item Aumentar la satisfacción de la ciudadanía en relación al estado de las carreteras de su municipio
\end{itemize}
\end{abstract}

\newpage
\tableofcontents{Contenido}
\newpage

\section{Introducción}

\subsection{Motivación y Objetivos}

TODO

\subsection{Tecnologías}

TODO

\subsection{Estructura del trabajo}

TODO

\section{Análisis y Diseño}

\subsection{Análisis y Definición de requisitos}

TODO

\subsection{Arquitectura}

TODO

\subsection{Diseño}

TODO

\subsection{Alcance del proyecto}

TODO


\section{Datos}

\subsection{Descripción de las fuentes de datos a utilizar}

TODO

\subsection{Estudio de los datos (reporting)}

TODO

\subsection{Limpieza y normalización de los datos}

TODO


\section{Técnicas de (Deep) Machine Learning y métodos de evaluación}

TODO


\subsection{Explicar las técnicas de ML que se van a utilizar en el proyecto}

TODO

\subsection{Explicar los métodos de evaluación que se van a utilizar en el proyecto}

TODO


\section{Implementación y evaluación de las técnicas}

\subsection{Detalles de la implementación de las técnicas de ML aplicadas}

TODO

\subsection{Evaluación de las técnicas}

TODO


\section{Resultados}

\subsection{Persistencia y almacenado de datos}

TODO

\subsection{Agente (ML)}

TODO

\subsection{Visualización}

TODO

\subsection{Resultados del proyecto}

TODO


\section{Conclusiones}

\subsection{Evaluación del proyecto}

TODO

\subsection{Alternativas y posibles mejoras que podrían haberse aplicado al proyecto (trabajos futuros)}

TODO

\subsection{Conclusiones personales}

TODO

\end{document}

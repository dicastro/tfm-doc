\begin{abstract}
El estado del asfalto en carreteras tanto de ámbito nacional como de ámbito urbano es de alta importancia en relación a la seguridad vial. En la actualidad, gracias a los avances tecnológicos, se están desarrollando sistemas de detección automática de baches en el asfalto, permitiendo una detección precoz de estas irregularidades en el asfalto.

Este proyecto pretende contribuir en este ámbito desarrollando un sistema de detección automática de baches a partir de imágenes. Partiendo de un conjunto de imágenes etiquetadas con baches, se ha entrenado una red neuronal YOLO v3 y otra YOLO v3 Tiny. El entrenamiento se ha realizado con distintos tamaños de red y con distintos subconjuntos de imágenes, dando lugar a quince modelos. Tras un estudio comparativo de las precisiones de los modelos, aquellos con mejores resultados han sido exportados y transformados para ser ejecutables en un dispositivo móvil. Finalmente se ha desarrollado una aplicación móvil Android que carga los modelos anteriormente exportados y los ejecuta con imágenes obtenidas de la cámara del dispositivo.

Todo el preprocesamiento de las imágenes, entrenamiento de las redes neuronales, evaluación, exportación y transformación de los modelos se ha realizado en la nube en la plataforma de Google Colab.
\end{abstract}
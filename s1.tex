\section{Introducción}

\subsection{Motivación}

En los últimos años las prioridades en relación a la seguridad vial han provocado un cambio de mentalidad en la sociedad.  Muchos de los últimos avances tecnológicos están orientados al desarrollo de medios de transporte más seguros. Uno de estos avances que contribuye a la mejora de la seguridad es la inclusión de un sistema de detección de desperfectos en la calzada.

En la actualidad, en el ayuntamiento de Madrid, existe un sistema de sugerencias y reclamaciones \cite{s1_syr} que permite al ciudadano, entre otras opciones, denunciar la existencia de irregularidades en el asfalto, tener un registro de las mismas y planificar su subsanación. Según el informe de sugerencias y reclamaciones del ayuntamiento de Madrid \cite{s1_syrreport}, en el primer semestre del año, se registraron 3.000 reclamaciones en materia de \textit{vías y espacios públicos} de las cuales el 50\% correspondieron a la submateria \textit{aceras y calzadas}.

Este sistema de funcionamiento actual, que delega en el ciudadano la tarea de reporte de este tipo de desperfectos, dificulta y retrasa la puesta en conocimiento de las irregularidades a las autoridades responsables aumentando la probabilidad de que suceda algún incidente.

Algunas marcas de vehículos han desarrollado innovaciones, que son capaces de detectar baches cuando pasan sobre ellos y adaptar la dureza de la suspensión para conseguir una conducción más segura y cómoda. También contemplan un envío sistemático de la detección de baches, en tiempo real, tanto a otros vehículos como a las autoridades pertinentes. Otras marcas plantean la inclusión de cámaras que permitan la detección del bache sin necesidad de pasar por encima de este.

Con estas innovaciones se recupera el control sobre la detección de baches, liberando al ciudadano de esta tarea y reduciendo la probabilidad de incidentes gracias a disponer de la información con más antelación.

% (porque has hecho este proyecto y que objetivos persigues)
% motivación: aplicar los conocimientos adquiridos durante el master en un proyecto "real"
% adquirir/complementar conocimientos que faltan en el máster

\subsection{Objetivos}

Este trabajo trata de desarrollar un sistema de detección de baches en tiempo real en línea con los últimos avances tecnológicos. El objetivo es entrenar una red neuronal con un conjunto de imágenes donde los baches han sido etiquetados previamente y obtener como resultado un modelo exportable para ser ejecutado en un dispositivo móvil.

Gracias a este trabajo he podido ampliar los conocimientos adquiridos durante el máster de Big Data \& Data Science, impartido por la U-TAD, centrando el proyecto en el procesamiento de imágenes y tratando al mimso tiempo que tenga una utilidad social.

\subsection{Estructura del trabajo}

La sección \textit{\nameref{sec:estado_del_arte}} explica cómo se pueden solucionar este tipo de problemas, y se centra en la resolución de los mismos mediante el uso de redes neuronales, valorando los pros y contras de cada uno de ellas.

La sección \textit{\nameref{sec:definicion_de_requisitos_y_analisis}} realiza un análisis de requisitos del proyecto y plantea una arquitectura para su desarrollo.

La sección \textit{\nameref{sec:datos}} describe el conjunto de datos utilizado, el análisis exploratorio que se ha realizado sobre el mismo y el preprocesamiento que se realiza sobre los datos antes de ser utilizados.

La sección \textit{\nameref{sec:tecnicas_de_deep_learning_y_metodos_de_evaluacion}} realiza una descripción a nivel teórico de las principales técnicas y conceptos que se aplican en el tipo de red neuronal seleccionada para el desarrollo del proyecto. También describe las métricas que se utilizan para evaluar los modelos generados.

La sección \textit{\nameref{sec:implementacion_y_evaluacion_de_las_tecnicas}} describe cómo se ha implementado la red neuronal y el resultado de la evaluación de los modelos generados.

La sección \textit{\nameref{sec:resultados}} muestra algunos ejemplos concretos obtenidos con los distintos modelos generados.

La sección \textit{\nameref{sec:conclusiones}} contiene una valoración del proyecto y se comentan posibles mejoras a realizar.

% (este punto es pura formalidad. Se suele poner uno o dos párrafos explicando la estructura del proyecto como hacen en los libros)
% en el tema X se hablará de tatata, en el tema Y se hablará de blablabla